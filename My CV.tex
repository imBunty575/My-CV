%%%%%%%%%%%%%%%%%%%%%%%%%%%%%%%%%%%%%%%%%
% Medium Length Professional CV
% LaTeX Template
% Version 2.0 (8/5/13)
%
% This template has been downloaded from:
% http://www.LaTeXTemplates.com
%
% Original author:
% Rishi Shah 
%
% Important note:
% This template requires the resume.cls file to be in the same directory as the
% .tex file. The resume.cls file provides the resume style used for structuring the
% document.
%
%%%%%%%%%%%%%%%%%%%%%%%%%%%%%%%%%%%%%%%%%

%----------------------------------------------------------------------------------------
%	PACKAGES AND OTHER DOCUMENT CONFIGURATIONS
%----------------------------------------------------------------------------------------

\documentclass{resume} % Use the custom resume.cls style

\usepackage{fontawesome}
\usepackage{graphicx}
\usepackage{hyperref}
\usepackage{array}
\usepackage{multicol}
\usepackage{tikz}
\usepackage[left=0.75in,top=0.75in,right=0.75in,bottom=0.75in]{geometry} % Document margins
\newcommand{\tab}[1]{\hspace{.2667\textwidth}\rlap{#1}}
\newcommand{\itab}[1]{\hspace{0em}\rlap{#1}}


\name{Satyajeet Moharana}

\begin{document}
\begin{center}
    \vspace{-1cm}
    \large{Cuttack, Odisha, India - 754105} \\ % Your address
    {\faPhone} \ \texttt{+91 9348690612} \\ % Your phone number
    {\faEnvelope} \ \texttt{\href{mailto:satyajeet@kasi.re.kr}{satyajeet@kasi.re.kr}} \\
    {\faEnvelope} \ \texttt{\href{mailto:satyajeetbunty5752017@gmail.com}{satyajeetbunty5752017@gmail.com}} % Your email
\end{center}


%----------------------------------------------------------------------------------------
%	EDUCATION SECTION
%----------------------------------------------------------------------------------------
\begin{rSection}{Research Interest}
\centering
{\bf Star Formation and Evolution – Stellar Chemical Compositions}
\end{rSection}

\begin{rSection}{Education}
{\bf Korea Astronomy and Space Science Institute, South Korea} \hfill {\em \bf Sept 2024 - Present}\\
\textit{Doctoral Student}\\
\textit{Supervisor:} Prof. Chang Won Lee \\
\textit{Thesis:} Understanding on early processes in stellar/substellar
formation in dense cores and filaments

{\bf Indian Institute of Astrophysics, Bengaluru} \hfill {\em \bf June 2023 - Aug 2024}\\
\textit{Master's Dissertation Student}\\
\textit{Guide:} Prof. Gajendra Pandey\\
\textit{Thesis:} Helium Abundance of the Sun: A Spectroscopic Analysis

{\bf Indian Institute of Science Education and Research Berhampur} \hfill {\em \bf Aug 2019 - July 2024} \\
\textit{BS-MS Dual Degree (Integrated Bachelors and Masters)}
\\ \textit{Major:} Physical Science; \textit{Minor:} Computer Science\\
\textit{Cumulative Point Index (CPI):} \textbf{9.4/10}

{\bf B.J.B. Higher Secondary School, Bhubaneswar} \hfill {\em \bf June 2017 - May 2019} \\
\textit{Higher Secondary Education} 
\\ \textit{Percentage:} \textbf{91.67}

{\bf Baleswar High School, Balada} \hfill{\em \bf May 2017}\\
\textit{Secondary Education}
\\ \textit{Percentage:} \textbf{95.83}
\end{rSection}

\begin{rSection}{Publications/ Posters}
\begin{enumerate}
    \item[\textbf{1.}] {\bf Helium Abundance of the Sun: A Spectroscopic Analysis} \hfill {\em \bf \textit{ApJ}} \\
    \textbf{\textcolor{blue}{\textit{Moharana S}}}\textit{, B. P. Hema, Pandey G. (2024)}\\
    \textbf{IOP: }\href{https://doi.org/10.3847/1538-4357/ad6ccf}{https://doi.org/10.3847/1538-4357/ad6ccf}
    
    \item[\textbf{2.}] {\bf Helium Abundance of the Sun: A Spectroscopic Analysis} \hfill {\em \textbf{ASI - 2024 (Poster)}} \\
    \textbf{\textcolor{blue}{\textit{Moharana S}}}\textit{, Pandey G., B. P. Hema (2024)} \hfill
    {\em Feb 2024, IISc Bengaluru}\\
    \textbf{Bibcode:} \href{https://ui.adsabs.harvard.edu/abs/2024asi..confP.169M/abstract}{2024asi..confP.169M} \\
    Poster presentation at the 42$^{\rm nd}$ Annual Meeting of the Astronomical Society of India - 2024.
\end{enumerate}

\end{rSection}
%--------------------------------------------------------------------------------
%    Projects And Seminars
%-----------------------------------------------------------------------------------------------
\begin{rSection}{Research Experience}
\begin{enumerate}
\item[\textbf{1.}] {\bf Filaments and dense cores in Aquila molecular cloud} \hfill {\em \bf Sep 2024 - Present}\\
    \textit{PhD, Guide: Prof. Chang Won Lee, KASI}
    \begin{itemize}
    \itemsep -5pt
        \item Study about the filaments and dense cores in the Aquila molecular cloud complex, focusing on their physical and chemical properties
        \item Analysis of gas kinematics between filaments and embedded cores to understand the star formation process.
        % \item 
    \end{itemize}
\vspace{0.3cm}
    \item[\textbf{2.}] {\bf Hydrogen-deficient Carbon (HdC) stars} \hfill {\em \bf Jan 2024 - Aug 2024}\\
    \textit{MS Thesis, Guide: Prof. Gajendra Pandey, IIA Bengaluru}
    \begin{itemize}
    \itemsep -5pt
        \item Investigation of HdC stars in the Northern Hemisphere using low-resolution spectroscopy with the TANSPEC on the Devasthal Optical Telescope, India.
        \item Reduction of data obtained from the observational campaigns targeting the potential HdCs.
        \item Determination of carbon and oxygen isotopic ratios from rovibrational band heads of $^{12}$C$^{16}$O, $^{13}$C$^{16}$O, and $^{12}$C$^{18}$O spectral lines in the K-band to identify HdC star candidates.
    \end{itemize}
\vspace{0.3cm}
    \item[\textbf{3.}] {\bf Solar photospheric helium abundance} \hfill {\em \bf June 2023 - Jan 2024}\\
    \textit{MS Thesis, Guide: Prof. Gajendra Pandey, IIA Bengaluru}
    \begin{itemize}
    \itemsep -5pt
        \item Spectroscopic analysis of the solar photosphere.
        \item Abundance analyses of magnesium and carbon utilizing MgH molecular and Mg I atomic lines alongside CH and C$_{2}$ Swan molecular and C I atomic lines for various He/H ratios.
        \item The solar He/H ratio was determined by enforcing that for an adopted model atmosphere (1D LTE) with an appropriate He/H ratio, the species' abundance derived from its atomic and molecular hydride lines should be the same within the margin of error.
    \end{itemize}
\vspace{0.3cm}
    \item[\textbf{4.}] {\bf Numerical Modeling and Simulation of DNA Detection using \hfill {\em \bf Jan - April 2023}\\Graphene Field Effect Transistors} \\
    \textit{Course work Project for PHY312: Numerical Methods and Programming\\Guide: Prof. Achanta Venugopal, TIFR Mumbai}
    \begin{itemize}
    \itemsep -5pt
        \item Simulated and analyzed DNA detection using Graphene Field-Effect Transistors (GFETs) through \textit{Monte Carlo} simulation.
        \item Investigated electrostatic potential distribution and drain-source current for DNA sensing.
        \item Visualized potential distribution and IDS-VG/IDS-VDS characteristics, noting surface voltage distortion with DNA molecule detection.
        
    \end{itemize}
\vspace{0.3cm}
    \item[\textbf{5.}] {\bf Describing Wavefunction of Nucleons inside a Spherical Nucleus} \hfill {\em \bf Jan - April 2023}\\
    \textit{Course work Project for PHY312: Numerical Methods and Programming\\Guide: Prof. Achanta Venugopal, TIFR Mumbai}
    \begin{itemize}
    \itemsep -5pt
        \item Solved and plotted wavefunctions and spherical Bessel functions using the \textit{Newton Raphson} method, offering insights into wavefunction behavior.
        \item Developed adaptable code for higher \( l \) values and lower tolerances, enhancing numerical method applications in physics.
    \end{itemize}
\vspace{0.3cm}
    \item[\textbf{6.}] {\bf The Atmospheres of Red Giants and their Chemical Composition} \hfill {\em \bf May - July 2022}\\
    \textit{Summer Internship, Guide: Prof. Gajendra Pandey, IIA Bengaluru}
    \begin{itemize}
    \itemsep -5pt
        \item Understanding the atmospheres of red giants and their chemical composition.
        \item Acquiring Spectroscopic Data Reduction Techniques using IRAF software.
    \end{itemize}
\vspace{0.3cm}
    \item[\textbf{7.}] {\bf IPL Score Prediction Using Deep Learning} \hfill {\em \bf Jan - April 2022}\\
    \textit{Course work Project for IDC302: Introduction to Data Science II\\Guide: Prof. P. Radhakrishna, NIT Warangal}
    \begin{itemize}
        \item Predicted the scorecards of upcoming matches of Indian Premier League - 2022 using a deep learning algorithm with early stopping, based on historical team and player performance data.
    \end{itemize}
\vspace{0.3cm}
    \item[\textbf{8.}] {\bf IPL Analysis with Python} \hfill {\em \bf Aug - Nov 2021}\\
    \textit{Course work Project for IDC301: Introduction to Data Science I\\Guide: Prof. P. Radhakrishna, NIT Warangal}
    \begin{itemize}
        \item Conducted statistical analysis of individual player and team performance in IPL matches up to the 2021 season using a machine learning algorithm, Principal Component Analysis (PCA).
    \end{itemize}
\end{enumerate}



\end{rSection}
%----------------------------------------------------------------------------------------
%	TECHNICAL STRENGTHS SECTION
%----------------------------------------------------------------------------------------

\begin{rSection}{Technical Skills and Experiences}
\begin{enumerate}
    \item[\textbf{1.}] {\bf Computational Skills}
    \begin{itemize}
    \itemsep -3pt
        \item \textit{Programming Languages:} Python, MATLAB, Java, C, \LaTeX, HTML \& CSS, SQL
        \item \textit{Operating Systems:} Linux environments, Microsoft, MacOS
        \item \textit{Softwares:} CARTA, DS9, IRAF, MOOG, TOPCAT, pyTANSPEC pipeline
    \end{itemize}
    %\vspace{0.3cm}
    \item[\textbf{2.}] {\bf Language Proficiency}
    \begin{itemize}
    \itemsep -3pt
        \item Odia (Native language)
        \item Hindi (Major spoken language in India)
        \item English (Scored 7.5/10 in IELTS - April 2024)
    \end{itemize}
\end{enumerate}


\end{rSection}

%----------------------------------------------------------------------------------------
%	WORK EXPERIENCE SECTION
%----------------------------------------------------------------------------------------




%	EXAMPLE SECTION
%----------------------------------------------------------------------------------------

\begin{rSection}{Presentations/ Science Camps/ Workshops}
\begin{enumerate}
    \item[\textbf{1.}] {\bf UST-KASI Young Researchers' Workshop - 2024} \hfill {\em \bf Nov. 7-8, 2024}\\
    A two-day workshop for PhD students at KASI, designed to foster discussions on individual research projects and provide insights into the research environment in South Korea through interactions with professors and guest speakers. 
    
    \item[\textbf{2.}] {\bf Visiting Student's Programme Seminar} \hfill {\em \bf Aug. 28, 2024}\\
    \textit{Title: The Solar Helium Abundance} \hfill {\em \textit{IIA Bengaluru}}\\
    Presentation on my MS thesis work on spectroscopically determining the solar helium abundance.
    %\vspace{0.3cm}
    
    \item[\textbf{3.}] {\bf CrowdScience Student Talk Series} \hfill {\em \bf Sept. 8, 2022}\\
    \textit{Title: The Atmospheres of Red Giants and their Chemical Composition} \hfill{\em \textit{IISER Berhampur}}\\
    Presentation on my two-month summer internship work at IIA Bengaluru.
    %\vspace{0.3cm}
    
    \item[\textbf{4.}] {\bf National Science Camp (VIJYOSHI)} \hfill {\em \bf Dec. 5-7 2019}\\
    The top 1\% students at the bachelor's level in India were invited to the camp. \hfill {\em \textit{IISER Kolkata}}
\end{enumerate}


\end{rSection}

%----------------------------------------------------------------------------------------
% Extra Curricular
%----------------------------------------------------------------------------------------
\begin{rSection}{Teaching Experience}
\begin{enumerate}
\item[\textbf{1.}] {\bf Teaching Assistant, PHY201: Electromagnetism} \hfill {\em \bf Aug - Nov 2021}\\
\textit{Course Instructor: Prof. S.N. Mishra, Dept. of Physical Sciences, IISER Berhampur}\\
Course offered by Dept. of Physical Sciences at IISER Berhampur for 2$^{nd}$ year BS-MS students.
%\vspace{0.3cm}

\item[\textbf{2.}] {\bf Delivered lectures on high school level Mathematics} \hfill {\em \bf Jan 2021}\\
\textit{Course Instructor: Mr. Bishwanath Ojha, Saraswati Sishu Vidya Mandir, Cuttack, Odisha}\\
These lectures on Mathematics were delivered to the students of class 10 appearing in their High School Certificate Examination - 2021, Saraswati Sishu Vidya Mandir, Cuttack, Odisha.
\end{enumerate}

\end{rSection}

\begin{rSection}{Achievements/ Awards/ Participations}
\begin{itemize}
\itemsep -3pt
    \item Offered a PhD position at Korea Astronomy and Space Science Institute (\textcolor{blue}{KASI}), South Korea, starting in Fall 2024.
    \item Qualified \textcolor{blue}{Graduate Aptitude Test in Engineering (GATE) - 2024} exam in Physics.
    \item Secured 3$^{rd}$ position in Space Quiz Competition of \textcolor{blue}{National Students' Space Challenge (NSSC) - 2023} organized by IIT Kharagpur.
    \item Participated in \textcolor{blue}{Inter IISER Cultural Meet (IICM) - 2022} in Drama organized by IISER Pune.
    \item Participated in \textcolor{blue}{Inter IISER Sports Meet (IISM) - 2022} in Cricket and Athletics organized by IISER Bhopal.
    \item Selected for \textcolor{blue}{IASc-INSA-NASI Summer Research Fellowship Program - 2022}.
    \item Participated in the \textcolor{blue}{International Asteroid Hunt Campaign - 2021}.
    \item Recipient of \textcolor{blue}{Smt. and Shri G.A.S. Charyulu Memorial Award - 2020} for securing the highest Semester Point Index in the 2$^{nd}$ semester at IISER Berhampur.
    \item \textcolor{blue}{DST-INSPIRE} Scholar (2019 - 2024), a scholarship awarded to top 1\% students at the bachelor's level in India.
    \item Qualified \textcolor{blue}{IISER Aptitude Test - 2024} to join IISER Berhampur.
    \item 8$^{th}$ Rank in \textcolor{blue}{Higher Secondary Examination - 2019} (Council of Higher Secondary Education, Odisha).
    \item Represented India in \textcolor{blue}{SAKURA Science Plan for High School Students - 2018} (Japan - Asia Youth Exchange Program) in Tokyo, Japan.
    \item 7$^{th}$ Rank in \textcolor{blue}{Secondary Examination - 2017} (Board of Secondary Education, Odisha).
\end{itemize}
\end{rSection}

\begin{rSection}{Community Outreach/ Cultural/ Volunteering Activities}
    \begin{itemize}
    \itemsep -3pt
        \item Social media handler for \textcolor{blue}{Asteroid Day - 2024}, organized by Science COmmunication, Public outreach, and Education (SCOPE) at IIA Bengaluru.
        \item Science demonstrator in \textcolor{blue}{Zero Shadow Day - 2024}, conducted by SCOPE, IIA Bengaluru.
        \item Event Coordinator in \textcolor{blue}{National Science Day - 2024}, hosted by SCOPE, IIA Bengaluru.
        \item Volunteered in various \textcolor{blue}{telescope sessions} organized by SCOPE, IIA Bengaluru, and NAXATRA (Astronomy Club), IISER Berhampur.
        \item Event Coordinator in \textcolor{blue}{KYRAT - 2022} and \textcolor{blue}{ALORA - 2023}, the cultural fests of IISER Berhampur.
        \item Event Coordinator in \textcolor{blue}{LAZZAT - 2023}, the food fest of IISER Berhampur.
        \item Demonstrated various scientific experiments to High School Students as a part of \textcolor{blue}{Science To Reach Everyone And Motivate (STREAM) - 2022}, the science fest of IISER Berhampur.
        \item Directed a skit at Gopalpur Sea beach to raise awareness about cleanliness and personal hygiene as part of \textcolor{blue}{\textit{Swachh Bharat Campaign - 2021}}, Govt. of India.
        
    \end{itemize}
\end{rSection}

\begin{rSection}{Hobbies}

\begin{itemize}
\itemsep -3pt
    \item Playing Cricket, Badminton, Table tennis
    \item Listening to music
    \item Book reading
\end{itemize}
    
\end{rSection}

\begin{rSection}{Referee Details}
\begin{multicols}{2}
\begin{enumerate}
    \item[\textbf{1.}] {\bf Prof. Gajendra Pandey}\\
    Professor,\\
    Indian Institute of Astrophysics, Bengaluru\\
    {\faEnvelope} \ \texttt{\href{mailto:pandey@iiap.res.in}{pandey@iiap.res.in}}

    \item[\textbf{2.}] {\bf Dr. Hema B. P.}\\
    DST Women Scientist - A,\\
    Indian Institute of Astrophysics, Bengaluru\\
    {\faEnvelope} \ \texttt{\href{mailto:hema.bp@iiap.res.in}{hema.bp@iiap.res.in}}

    % \item[\textbf{3.}] {\bf Dr. A. Bala Sudhakara Reddy}\\
    % Assistant Professor,\\
    % Indian Institute of Astrophysics, Bengaluru\\
    % {\faEnvelope} \ \texttt{\href{mailto:balasudhakara.arumalla@iiap.res.in}{balasudhakara.arumalla@iiap.res.in}}

    % \item[\textbf{4.}] {\bf Dr. Vyas Akondi}\\
    % Assistant Professor,\\
    % Indian Institute of Science Education and \mbox{Research}, Berhampur\\
    % {\faEnvelope} \ \texttt{\href{mailto:vakondi@iiserbpr.ac.in}{vakondi@iiserbpr.ac.in}}
\end{enumerate}

\end{multicols}
\end{rSection}
\end{document}
